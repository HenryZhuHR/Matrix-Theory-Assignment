\chapter{LU分解}

\section{LU分解的Matlab实现}
用Matlab自带的函数\lstinline|rand()|创建一个大小为$5\times 5$的随机矩阵


\begin{lstlisting}[language=Matlab]  
>> A=rand(5,5)*20
A =
    14.2243   8.4833    0.5844    4.7457    4.6319
    4.4349   10.1572   18.5771    9.1770    9.7780
    2.3484    1.7103   14.6066   19.2618   12.4812
    5.9335    5.2496    9.7722   10.9361   13.5827
    6.3756   16.0203   11.5705   10.4227    7.9103
\end{lstlisting}


向下取整后作为项目测试用的矩阵$A$
\begin{equation}
    A=\begin{bmatrix}
        14 & 8  & 0  & 4  & 4  \\
        4  & 10 & 18 & 9  & 9  \\
        2  & 1  & 14 & 19 & 12 \\
        5  & 5  & 9  & 10 & 13 \\
        6  & 16 & 11 & 10 & 7
    \end{bmatrix}
\end{equation}

设置矩阵$A$
\begin{lstlisting}[language=Matlab]  
>> A = [14, 8, 0, 4, 4; 4, 10, 18, 9, 9;
        2, 1, 14, 19, 12; 5, 5, 9, 10, 13; 6, 16, 11, 10, 7]
A =
    14     8     0     4     4
     4    10    18     9     9
     2     1    14    19    12
     5     5     9    10    13
     6    16    11    10     7
\end{lstlisting}

调用Matlab自带的函数\lstinline|lu()|进行矩阵分解
\begin{lstlisting}[language=Matlab]  
>> [L,U] = lu(A)
L =

    1.0000         0         0         0         0
    0.2857    0.6136    0.7965    1.0000         0
    0.1429   -0.0114    1.0000         0         0
    0.3571    0.1705    0.5044    0.1823    1.0000
    0.4286    1.0000         0         0         0
U =
    14.0000    8.0000         0    4.0000    4.0000
          0   12.5714   11.0000    8.2857    5.2857
          0         0   14.1250   18.5227   11.4886
          0         0         0  -11.9799   -4.5366
          0         0         0         0    5.7024
\end{lstlisting}

代码可以在 {\url{https://github.com/whutug/whu-thesis}} 找到