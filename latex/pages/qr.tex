\chapter{QR 分解}
\section{QR 分解简介}
$QR$ 分解能够将矩阵 $A$ 分解成为一个 $m\times n$ 上三角矩阵 $R$ 和一个 $m\times m$ 正交矩阵 $Q$

\begin{equation}
    A = QR
    \label{QR-decomposition}
\end{equation}

另一种 $QR$ 分解的形式是引入 $n\times n$ 置换矩阵 $P$ 使得 $AP$ 分解成为一个 $m\times n$ 上三角矩阵 $R$ 和一个 $m\times m$ 正交矩阵 $Q$
\begin{equation}
    AP = QR
    \label{QRP-decomposition}
\end{equation}

\section{QR 分解的 Matlab 实现及结果验证}
用 Matlab 自带的函数 \lstinline|rand()| 创建一个大小为 $3\times 5$ 的随机矩阵

\begin{lstlisting}[language=Matlab]  
>> A=rand(3,5)*20
A =
    16.2945   18.2675    5.5700   19.2978   19.1433
    18.1158   12.6472   10.9376    3.1523    9.7075
     2.5397    1.9508   19.1501   19.4119   16.0056
\end{lstlisting}


向下取整后作为项目测试用的矩阵$A$
\begin{equation}
    A=\begin{bmatrix}
        16&   18&    5&   19&   19\\
        18&   12&   10&    3&    9\\
        2&    1&   19&   19&   16
    \end{bmatrix}
\end{equation}

% 随机产生列向量 $b$
% \begin{lstlisting}[language=Matlab]  
% >> b=rand(5,1)*20
% b =
%     14.7572
%     5.3824
%     8.4567
% \end{lstlisting}

% 取 $b=[14,5,8]$




设置矩阵$A$
\begin{lstlisting}[language=Matlab]  
>> A = [16, 18, 5, 19, 19;18, 12, 10, 3, 9;2, 1, 19, 19, 16];
\end{lstlisting}

调用 Matlab 自带的函数 \lstinline|qr()| 进行矩阵分解,\lstinline|[Q,R] = qr(A)| 函数可以对 $3\times 5$ 矩阵 $A$ 进行 $QR$ 分解,满足式\ref{QR-decomposition}

\begin{lstlisting}[language=Matlab]  
>> [Q, R] = qr(A)
Q =
   -0.6621    0.7481    0.0449
   -0.7448   -0.6502   -0.1497
   -0.0828   -0.1325    0.9877
R =
  -24.1661  -20.9384  -12.3313  -16.3866  -20.6074
         0    5.5301   -5.2799    9.7449    6.2410
         0         0   17.4946   19.1707   15.3096
\end{lstlisting}

为了验证结果,我们将 $3\times 3$ 的正交矩阵 $Q$ 和 $3\times 5$ 的上三角矩阵 $R$ 相乘得到结果矩阵 $RES$
\begin{lstlisting}[language=Matlab]  
>> RES = Q * R
RES =
        16.0000   18.0000    5.0000   19.0000   19.0000
        18.0000   12.0000   10.0000    3.0000    9.0000
         2.0000    1.0000   19.0000   19.0000   16.0000
\end{lstlisting}
得到的结果矩阵 $RES$ 与待分解矩阵 $A$ 一致


此外,该函数 \lstinline|[Q,R,P] = qr(A)| 还会额外返回一个 $5\times 5$ 的置换矩阵 $P$,并满足 
\begin{equation}
    AP = QR
\end{equation}
\begin{lstlisting}[language=Matlab]  
>> [Q, R, P] = qr(A)
Q =
   -0.7027   -0.2969   -0.6465
   -0.1110   -0.8519    0.5118
   -0.7027    0.4314    0.5657
R =
  -27.0370  -14.6466  -17.9754  -14.6836  -25.5945
         0  -19.2218   -1.8064  -15.1357   -6.4056
         0         0   12.6342   -4.9298    1.3739
P =
     0     1     0     0     0
     0     0     0     1     0
     0     0     1     0     0
     1     0     0     0     0
     0     0     0     0     1
\end{lstlisting}

为了验证结果,我们将
$3\times 5$ 的待分解矩阵 $A$ 和 $5\times 5$ 的置换矩阵 $P$ 相乘得到结果矩阵 $RES1$,并且将
$3\times 3$ 的正交矩阵 $Q$ 和 $3\times 5$ 的上三角矩阵 $R$ 相乘得到结果矩阵 $RES2$
\begin{lstlisting}[language=Matlab]  
>> RES1 = A * P
RES2 = Q * R
RES1 =
    19    16     5    18    19
     3    18    10    12     9
    19     2    19     1    16
RES2 =
   19.0000   16.0000    5.0000   18.0000   19.0000
    3.0000   18.0000   10.0000   12.0000    9.0000
   19.0000    2.0000   19.0000    1.0000   16.0000
\end{lstlisting}
矩阵 $RES1$ 和 矩阵 $RES2$ 的结果一致


\section{QR 分解的 C++ 实现及结果验证}
\begin{lstlisting}[language=C++]  
#include <iostream>
#include <Eigen/Dense>

using namespace std;
using namespace Eigen;
int main()
{
    MatrixXf A(3, 5);
    A << 16, 18,  5, 19, 19, 
         18, 12, 10,  3,  9, 
          2,  1, 19, 19, 16;
    cout << "matrix A:" << endl << A << endl << endl;

    HouseholderQR<MatrixXf> qr(A);
    qr.compute(A);
    MatrixXf R = qr.matrixQR().triangularView<Upper>();
    MatrixXf Q = qr.householderQ();
    cout << "matrix Q:" << endl << Q << endl << endl;
    cout << "matrix R:" << endl << R << endl << endl;

    return 0;
}    
\end{lstlisting}

编译运行后
\begin{lstlisting}[language=bash]  
E:\Projects\Matrix-Theory-Assignment\cpp\bin>qr.exe
matrix A:
16 18  5 19 19
18 12 10  3  9
 2  1 19 19 16

matrix Q:
 -0.662085   0.748083  0.0448963
 -0.744845  -0.650238  -0.149654
-0.0827606  -0.132525   0.987719

matrix R:
-24.1661 -20.9384 -12.3313 -16.3866 -20.6074
       0  5.53012 -5.27993  9.74489  6.24104
       0        0  17.4946  19.1707  15.3096
\end{lstlisting}
该结果与 Matlab 运行结果一致