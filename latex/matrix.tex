%% An Introduction to LaTeX Thesis Template of Wuhan University
%%
%% Created by WHUTUG

% 用来设置附录中代码的样式



\documentclass[type=master,oneside]{whu-thesis}
\whusetup
  {
    info               =
      {
        title          = {矩阵分解作业},
        title*         = {}, 
        student-number = {2021202120085},
        school         = {电子信息学院},
        author         = {朱鹤然},
        author*        = {Zhu Heran},
        subject        = {学科},
        major          = {电子信息学院},
        advisor        = {王文伟},
        direction      = {信息与通信工程},
        % date           = {2021/5},
        keywords       = {关键词 1 , 关键词 2 , 关键词 3 , 关键词 4 , 一个非常非常,非常非常长——的关键词 5},
        keywords*      = {key word 1 , key word 2 , key word 3 , key word 4 , {and a very very, very very long key word---the key word 5}},
      },
    style              =
      {
        graphics-path  = {{figures/}{data/}},
      },
    element            =
      {
        abstract       = {pages/abstract},
        bibliography   = {ref/refs},
        appendix       = {pages/appendix}
      }
  }

\usepackage{listings}  % 排代码用的宏包
\usepackage{color}
\definecolor{codegreen}{rgb}{0,0.6,0}
\definecolor{codegray}{rgb}{0.5,0.5,0.5}
\definecolor{codepurple}{rgb}{0.58,0,0.82}
\definecolor{backcolour}{rgb}{0.95,0.95,0.92}
\lstset{
    basicstyle=\footnotesize,
    keywordstyle      =   \color{codepurple}\bfseries,% 关键字颜色
    identifierstyle   =   \color{blue},
    commentstyle      =   \color{codegreen}% 注释颜色
    showstringspaces  =   false,% 不显示字符串内的空格
    backgroundcolor   =   \color[rgb]{0.98,0.98,0.98},% 背景颜色
    numbers           =   left,                               % 显示行号
}

\AtBeginEnvironment{lstlisting}{\linespread{1}}%放在导言区即可全文统一使用

\usepackage{hyperref}
\hypersetup{
	colorlinks=true,
	linkcolor=cyan,
	filecolor=blue,      
	urlcolor=red,
	citecolor=green,
}

\begin{document}
%%----------- 主体部分 ----------- %%
\chapter{LU分解}
用 Matlab 自带的函数 \lstinline|rand()| 创建一个大小为 $5\times 5$ 的随机矩阵


\begin{lstlisting}[language=Matlab]  
>> A=rand(5,5)*20
A =
    14.2243   8.4833    0.5844    4.7457    4.6319
    4.4349   10.1572   18.5771    9.1770    9.7780
    2.3484    1.7103   14.6066   19.2618   12.4812
    5.9335    5.2496    9.7722   10.9361   13.5827
    6.3756   16.0203   11.5705   10.4227    7.9103
\end{lstlisting}


向下取整后作为项目测试用的矩阵$A$
\begin{equation}
    A=\begin{bmatrix}
        14 & 8  & 0  & 4  & 4  \\
        4  & 10 & 18 & 9  & 9  \\
        2  & 1  & 14 & 19 & 12 \\
        5  & 5  & 9  & 10 & 13 \\
        6  & 16 & 11 & 10 & 7
    \end{bmatrix}
\end{equation}

随机产生列向量 $b$
\begin{lstlisting}[language=Matlab]  
>> b=rand(5,1)*20
b =
   12.9798
   16.0066
    9.0760
    8.6478
   16.5063
\end{lstlisting}

取 $b=[12,16,9,8,16]$

\section{LU 分解的 Matlab 实现及结果验证}
\subsection{矩阵的 LU 分解}
设置矩阵$A$
\begin{lstlisting}[language=Matlab]  
>> A = [14, 8, 0, 4, 4; 4, 10, 18, 9, 9;
        2, 1, 14, 19, 12; 5, 5, 9, 10, 13; 6, 16, 11, 10, 7]
A =
    14     8     0     4     4
     4    10    18     9     9
     2     1    14    19    12
     5     5     9    10    13
     6    16    11    10     7
\end{lstlisting}

调用 Matlab 自带的函数 \lstinline|lu()| 进行矩阵分解,\lstinline|[L,U] = lu(A)| 函数可以将满矩阵或稀疏矩阵 $A$ 分解为一个上三角矩阵 $U$ 和一个经过置换的下三角矩阵 $L$,使得
\begin{equation}
    A = LU
\end{equation}

\begin{lstlisting}[language=Matlab]  
>> [L,U] = lu(A)
L =

    1.0000         0         0         0         0
    0.2857    0.6136    0.7965    1.0000         0
    0.1429   -0.0114    1.0000         0         0
    0.3571    0.1705    0.5044    0.1823    1.0000
    0.4286    1.0000         0         0         0
U =
    14.0000    8.0000         0    4.0000    4.0000
          0   12.5714   11.0000    8.2857    5.2857
          0         0   14.1250   18.5227   11.4886
          0         0         0  -11.9799   -4.5366
          0         0         0         0    5.7024
\end{lstlisting}

为了验证结果,我们将上三角矩阵 $U$ 和下三角矩阵 $L$ 相乘
\begin{lstlisting}[language=Matlab]  
>> B=L*U
B =
   14.0000    8.0000         0    4.0000    4.0000
    4.0000   10.0000   18.0000    9.0000    9.0000
    2.0000    1.0000   14.0000   19.0000   12.0000
    5.0000    5.0000    9.0000   10.0000   13.0000
    6.0000   16.0000   11.0000   10.0000    7.0000
\end{lstlisting}
得到的结果矩阵 $B$ 与待分解矩阵一致


此外,该函数 \lstinline|[L,U,P] = lu(A)| 还可以返回一个置换矩阵 $P$,并满足 $A = P'LU$。在此语法中,$L$ 是单位下三角矩阵,$U$ 是上三角矩阵。
\begin{lstlisting}[language=Matlab]  
>> [L,U,P] = lu(A)
L =
    1.0000         0         0         0         0
    0.4286    1.0000         0         0         0
    0.1429   -0.0114    1.0000         0         0
    0.2857    0.6136    0.7965    1.0000         0
    0.3571    0.1705    0.5044    0.1823    1.0000

U =
   14.0000    8.0000         0    4.0000    4.0000
         0   12.5714   11.0000    8.2857    5.2857
         0         0   14.1250   18.5227   11.4886
         0         0         0  -11.9799   -4.5366
         0         0         0         0    5.7024

P =
     1     0     0     0     0
     0     0     0     0     1
     0     0     1     0     0
     0     1     0     0     0
     0     0     0     1     0
\end{lstlisting}

\begin{lstlisting}[language=Matlab]  
>> C=P'*L*U
C =
   14.0000    8.0000         0    4.0000    4.0000
    4.0000   10.0000   18.0000    9.0000    9.0000
    2.0000    1.0000   14.0000   19.0000   12.0000
    5.0000    5.0000    9.0000   10.0000   13.0000
    6.0000   16.0000   11.0000   10.0000    7.0000
\end{lstlisting}


\subsection{用 LU 分解对线性方程组求解}
假定需要求解的方程组为 $Ax=b$,即
\begin{equation}
    \begin{bmatrix}
        14 & 8  & 0  & 4  & 4  \\
        4  & 10 & 18 & 9  & 9  \\
        2  & 1  & 14 & 19 & 12 \\
        5  & 5  & 9  & 10 & 13 \\
        6  & 16 & 11 & 10 & 7
    \end{bmatrix}
    x = 
    \begin{bmatrix}
        12 \\ 16 \\ 9 \\ 8 \\ 16
    \end{bmatrix}
\end{equation}

求解上述方程组的过程如下
\begin{lstlisting}
>> b=[12;16;9;8;16];
>> [L,U,P] = lu(A);
>> y = L\(P*b)
y =
   12.0000
   10.8571
    7.4091
    0.0080
   -1.8752
>> x = U\y
x =
    0.7046
    0.3694
    0.6296
    0.1239
   -0.3288
\end{lstlisting}

验证结果
\begin{lstlisting}
>> D=A*x
D =
   12.0000
   16.0000
    9.0000
    8.0000
   16.0000
\end{lstlisting}


\section{LU 分解的 C++ 实现及结果验证}
\subsection{矩阵的 LU 分解}
\subsection{用 LU 分解对线性方程组求解}
\include{pages/QR}
\end{document}